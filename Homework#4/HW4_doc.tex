%%%%%%%%%%%%%%%%%%%%%%%%%%%%%%%%%%%%%%%%%
% Cleese Assignment (For Students)
% LaTeX Template
% Version 2.0 (27/5/2018)
%
% This template originates from:
% http://www.LaTeXTemplates.com
%
% Author:
% Vel (vel@LaTeXTemplates.com)
%
% License:
% CC BY-NC-SA 3.0 (http://creativecommons.org/licenses/by-nc-sa/3.0/)
% 
%%%%%%%%%%%%%%%%%%%%%%%%%%%%%%%%%%%%%%%%%

%----------------------------------------------------------------------------------------
%	PACKAGES AND OTHER DOCUMENT CONFIGURATIONS
%----------------------------------------------------------------------------------------

\documentclass[11pt]{article}

\input{structure.tex} % Include the file specifying the document structure and custom commands
\usepackage{ctex}
\usepackage{tabularx}
\usepackage{booktabs}
\usepackage{graphicx}
\usepackage{caption}
\usepackage{subcaption}
\usepackage{float}
%----------------------------------------------------------------------------------------
%	ASSIGNMENT INFORMATION
%----------------------------------------------------------------------------------------

% Required
\newcommand{\assignmentQuestionName}{Problem} % The word to be used as a prefix to question numbers; example alternatives: Problem, Exercise
\newcommand{\assignmentClass}{ZJU Computational Physics} % Course/class
\newcommand{\assignmentTitle}{Homework\ \#4} % Assignment title or name
\newcommand{\assignmentAuthorName}{NAKO} % Student name

% Optional (comment lines to remove)
%\newcommand{\assignmentClassInstructor}{} % Intructor name/time/description
\newcommand{\assignmentDueDate}{Monday,\ December\ 13,\ 2024\\
Github: https://github.com/NAKONAKO4/ZJU-computational-physics-NAKO} % Due date
\usepackage{listings, xcolor}
\lstdefinestyle{lfonts}{
  basicstyle   = \footnotesize\ttfamily,
  stringstyle  = \color{purple},
  keywordstyle = \color{blue!60!black}\bfseries,
  commentstyle = \color{olive}\scshape,
}
\lstdefinestyle{lnumbers}{
  numbers     = left,
  numberstyle = \tiny,
  numbersep   = 1em,
  firstnumber = 1,
  stepnumber  = 1,
}
\lstdefinestyle{llayout}{
  breaklines       = true,
  tabsize          = 2,
  columns          = flexible,
}
\lstdefinestyle{lgeometry}{
  xleftmargin      = 20pt,
  xrightmargin     = 0pt,
  frame            = tb,
  framesep         = \fboxsep,
  framexleftmargin = 20pt,
}
\lstdefinestyle{lgeneral}{
  style = lfonts,
  style = lnumbers,
  style = llayout,
  style = lgeometry,
}
\lstdefinestyle{python}{
  language = {Python},
  style    = lgeneral,
}
%----------------------------------------------------------------------------------------

\begin{document}

%----------------------------------------------------------------------------------------
%	TITLE PAGE
%----------------------------------------------------------------------------------------

\maketitle % Print the title page

\thispagestyle{empty} % Suppress headers and footers on the title page

\newpage

%----------------------------------------------------------------------------------------
%	QUESTION 1
%----------------------------------------------------------------------------------------

\begin{question}

\questiontext{Random Number Generator}
\answer{选取前十组数据进行随机数生成,采用线性同余生成器,数学公式表示为$X_{n+1} = (a \cdot X_n + c) \mod m$,生成2500个随机数。
测试方法采用四种,分别为:分布均匀性测试、自相关性测试、Kolmogorov-Smirnov 测试、频谱分析(FFT后观察各个频率分布是否均匀)
}
\begin{figure}[H]
  \centering
  \includegraphics[width=0.7\columnwidth]{1_1.png}
  \caption{在10组参数下生成的随机数分布的均匀性}
\end{figure}
\begin{figure}[H]
  \centering
  \includegraphics[width=0.7\columnwidth]{1_2.png}
  \caption{在10组参数下生成的随机数分布的自相关性}
\end{figure}

\begin{figure}[H]
  \centering
  \includegraphics[width=0.7\columnwidth]{1_4.png}
  \caption{在10组参数下生成的随机数分布的Kolmogorov-Smirnov 测试}
\end{figure}

\begin{figure}[H]
  \centering
  \includegraphics[width=0.7\columnwidth]{1_3.png}
  \caption{在10组参数下生成的随机数分布的频谱分析}
\end{figure}
\answer{可以看到,分布均匀性较好,自相关性基本为0,Kolmogorov-Smirnov 测试的stat 较小且 p-value 较大,说明随机数序列接近均匀分布,频谱分析可以看到十组数据在各个频率上分布接近均匀。
        事实上2500个随机数相对并不多,但由于在较多随机数生成时存在绘图问题,只进行了生成10000个随机数时的分布均匀性测试,得到结果为:
        \begin{center}
          \includegraphics[width=0.7\columnwidth]{1_1_2.png}
        \end{center}
        可以看到随着生成随机数越多,随机数生成器的均匀性能体现的越好,这说明随机数生成器的性能是好的,能够生成较随机的伪随机数。
}
\lstinputlisting[style = python]{1.py}
\end{question}

%----------------------------------------------------------------------------------------
%	QUESTION 2
%----------------------------------------------------------------------------------------

\begin{question}

\questiontext{Random walks in two dimensions}
\answer{代码运行结果为:
\begin{center}
  \includegraphics[width=\columnwidth]{2_1.png}
\end{center}

值得注意的是,在实验过程中我发现由于浮点误差,代码中的"is\_self\_avoiding()"函数会因为设置的随机行走方向存在浮点数,而导致判断错误得到有634步self-avoiding random walks,但这是错误的,
所以我修改随机行走方向全部为整数用来输入到"is\_non\_reversal()"和"is\_self\_avoiding()"函数中用来判断随机行走类型,然后在输入到"calculate\_statistics()"函数中进行x和y相关的计算时,对y数据乘以$\sqrt{3}$来将随机行走方向回到真实数值。
}

\lstinputlisting[style = python]{2.py}

%--------------------------------------------
%--------------------------------------------

\end{question}
%----------------------------------------------------------------------------------------
%	QUESTION 3
%----------------------------------------------------------------------------------------

\begin{question}
  \questiontext{Numerical solution of the potential within a rectangular region.}
  \answer{a. 对$n_x=n_y=9, n_x=n_y=45, n_x=n_y=72$进行分析,结果如下,其中$n_x=n_y=45$是为了与c题进行对比。
  \begin{center}
  \includegraphics[width=\columnwidth]{3_1.png}
  \end{center}}
  \lstinputlisting[style = python]{3_a.py}
  \answer{b. 结果如下,可以看到进行边界值平均作为初始猜测值的方法是更好的,需要的迭代次数远少于随机猜测初始值的方法。
          对两种方法得到的结果进行差值分析,可以看到两种方法的结果相差很小。
          \begin{center}
            \includegraphics[width=\columnwidth]{3_2_2.png}
          \end{center}
  } 
  \lstinputlisting[style = python]{3_b.py}
  \answer{c. 结果如下,可以看到比a中的$n_x=n_y=45$情况运算更快,迭代次数更少。
  \begin{center}
    \includegraphics[width=\columnwidth]{3_3.png}
  \end{center}
  }
  \lstinputlisting[style = python]{3_c.py}


\end{question}
\end{document}
