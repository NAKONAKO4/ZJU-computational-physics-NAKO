%%%%%%%%%%%%%%%%%%%%%%%%%%%%%%%%%%%%%%%%%
% Cleese Assignment (For Students)
% LaTeX Template
% Version 2.0 (27/5/2018)
%
% This template originates from:
% http://www.LaTeXTemplates.com
%
% Author:
% Vel (vel@LaTeXTemplates.com)
%
% License:
% CC BY-NC-SA 3.0 (http://creativecommons.org/licenses/by-nc-sa/3.0/)
% 
%%%%%%%%%%%%%%%%%%%%%%%%%%%%%%%%%%%%%%%%%

%----------------------------------------------------------------------------------------
%	PACKAGES AND OTHER DOCUMENT CONFIGURATIONS
%----------------------------------------------------------------------------------------

\documentclass[11pt]{article}

\input{structure.tex} % Include the file specifying the document structure and custom commands
\usepackage{ctex}
%----------------------------------------------------------------------------------------
%	ASSIGNMENT INFORMATION
%----------------------------------------------------------------------------------------

% Required
\newcommand{\assignmentQuestionName}{Question} % The word to be used as a prefix to question numbers; example alternatives: Problem, Exercise
\newcommand{\assignmentClass}{ZJU Computational Physics} % Course/class
\newcommand{\assignmentTitle}{Homework\ \#1} % Assignment title or name
\newcommand{\assignmentAuthorName}{GUAN Yifeng} % Student name

% Optional (comment lines to remove)
%\newcommand{\assignmentClassInstructor}{Jones 10:30am} % Intructor name/time/description
\newcommand{\assignmentDueDate}{Monday,\ October\ 18,\ 2024} % Due date

%----------------------------------------------------------------------------------------

\begin{document}

%----------------------------------------------------------------------------------------
%	TITLE PAGE
%----------------------------------------------------------------------------------------

\maketitle % Print the title page

\thispagestyle{empty} % Suppress headers and footers on the title page

\newpage

%----------------------------------------------------------------------------------------
%	QUESTION 1
%----------------------------------------------------------------------------------------

\begin{question}

\questiontext{The cooling coffee.}



\answer{a. 计算方法采用两种:第一种是简单地对每一个$\Delta t$求r值,之后对所有r值求均值;第二种是将所有数据在牛顿冷却函数下做最小二乘法拟合。
		由于时间间隔2min并不是一个足够小的值,所以第一种方法的误差会偏大,而采用最小二乘法拟合会更加精准。\par
		通过代码计算(参见/1/a.py)可得,由第一种方法计算得到的黑咖啡r值为0.0232min$^{-1}$,奶咖啡为0.0214min$^{-1}$;由第二种方法得到的黑咖啡r值为0.0259min$^{-1}$,奶咖啡为0.0237min$^{-1}$。之后的计算都采用第二种方法得到的结果。}
\answer{b. 参见/1/b.py,图像中可见仍然存在一定的与实际值的误差。
		\begin{center}
			\includegraphics[width=0.7\columnwidth]{b.png}
		\end{center}}
\answer{c. 设置两个新的步长4min和1min,其中4min步长通过直接取数据中的每个4min时间间隔即可,1min步长通过差值实现。}
\answer{d. 降温到49度需要27.54min,降温到33度需要54.30分钟,降温到25度需要81.06分钟。这说明物体与周围环境温差越小降温越慢。}
\answer{e. 牛顿冷却定律并不完全适合这个问题,还存在液体的蒸发所带走的热量,以及液体表面积和与环境接触材质的影响,需要考虑物体与环境的热传导系数、物体的传热方式来进行优化。}
\end{question}

%----------------------------------------------------------------------------------------
%	QUESTION 2
%----------------------------------------------------------------------------------------

\begin{question}

\questiontext{How much wood would a woodchuck chuck if a woodchuck could chuck wood?}

%--------------------------------------------

\begin{subquestion}{Suppose ``chuck" implies throwing.} % Subquestion within question

\answer{According to the Associated Press (1988), a New York Fish and Wildlife technician named Richard Thomas calculated the volume of dirt in a typical 25--30 foot (7.6--9.1 m) long woodchuck burrow and had determined that if the woodchuck had moved an equivalent volume of wood, it could move ``about \textbf{700 pounds (320 kg)} on a good day, with the wind at his back".}

\end{subquestion}

%--------------------------------------------

\begin{subquestion}{Suppose ``chuck" implies vomiting.} % Subquestion within question

\answer{A woodchuck can ingest 361.92 cm\textsuperscript{3} (22.09 cu in) of wood per day. Assuming immediate expulsion on ingestion with a 5\% retainment rate, a woodchuck could chuck \textbf{343.82 cm\textsuperscript{3}} of wood per day.}

\end{subquestion}

%--------------------------------------------

\end{question}

%----------------------------------------------------------------------------------------
%	QUESTION 3
%----------------------------------------------------------------------------------------

\begin{question}

\questiontext{Identify the author of Equation \ref{eq:bayes} below and briefly describe it in English.}

\begin{equation}\label{eq:bayes}
	P(A|B) = \frac{P(B|A)P(A)}{P(B)}
\end{equation}

\answer{Lorem ipsum dolor sit amet, consectetur adipiscing elit. Praesent porttitor arcu luctus, imperdiet urna iaculis, mattis eros. Pellentesque iaculis odio vel nisl ullamcorper, nec faucibus ipsum molestie. Sed dictum nisl non aliquet porttitor. Etiam vulputate arcu dignissim, finibus sem et, viverra nisl. Aenean luctus congue massa, ut laoreet metus ornare in. Nunc fermentum nisi imperdiet lectus tincidunt vestibulum at ac elit. Nulla mattis nisl eu malesuada suscipit.}

\end{question}

%----------------------------------------------------------------------------------------

\assignmentSection{Bonus Questions}

%----------------------------------------------------------------------------------------
%	QUESTION 4
%----------------------------------------------------------------------------------------

\begin{question}

\questiontext{The table below shows the nutritional consistencies of two sausage types. Explain their relative differences given what you know about daily adult nutritional recommendations.}

\begin{table}[h]
	\centering % Centre the table
	\begin{tabular}{l l l}
		\toprule
		\textit{Per 50g} & Pork & Soy \\
		\midrule
		Energy & 760kJ & 538kJ\\
		Protein & 7.0g & 9.3g\\
		Carbohydrate & 0.0g & 4.9g\\
		Fat & 16.8g & 9.1g\\
		Sodium & 0.4g & 0.4g\\
		Fibre & 0.0g & 1.4g\\
		\bottomrule
	\end{tabular}
\end{table}

\answer{Lorem ipsum dolor sit amet, consectetur adipiscing elit. Praesent porttitor arcu luctus, imperdiet urna iaculis, mattis eros. Pellentesque iaculis odio vel nisl ullamcorper, nec faucibus ipsum molestie. Sed dictum nisl non aliquet porttitor. Etiam vulputate arcu dignissim, finibus sem et, viverra nisl. Aenean luctus congue massa, ut laoreet metus ornare in. Nunc fermentum nisi imperdiet lectus tincidunt vestibulum at ac elit. Nulla mattis nisl eu malesuada suscipit.}

\end{question}

%----------------------------------------------------------------------------------------
%	QUESTION 5
%----------------------------------------------------------------------------------------

\begin{question}

\lstinputlisting[
	caption=Luftballons Perl Script, % Caption above the listing
	label=lst:luftballons, % Label for referencing this listing
	language=Perl, % Use Perl functions/syntax highlighting
	frame=single, % Frame around the code listing
	showstringspaces=false, % Don't put marks in string spaces
	numbers=left, % Line numbers on left
	numberstyle=\tiny, % Line numbers styling
	]{luftballons.pl}

%--------------------------------------------

\begin{subquestion}{How many luftballons will be output by the Listing \ref{lst:luftballons} above?} % Subquestion within question

\answer{99 luftballons.}

\end{subquestion}

%--------------------------------------------

\begin{subquestion}{Identify the regular expression in Listing \ref{lst:luftballons} and explain how it relates to the anti-war sentiments found in the rest of the script.} % Subquestion within question

\answer{Lorem ipsum dolor sit amet, consectetur adipiscing elit. Praesent porttitor arcu luctus, imperdiet urna iaculis, mattis eros. Pellentesque iaculis odio vel nisl ullamcorper, nec faucibus ipsum molestie. Sed dictum nisl non aliquet porttitor. Etiam vulputate arcu dignissim, finibus sem et, viverra nisl. Aenean luctus congue massa, ut laoreet metus ornare in. Nunc fermentum nisi imperdiet lectus tincidunt vestibulum at ac elit. Nulla mattis nisl eu malesuada suscipit.}

\end{subquestion}

%--------------------------------------------

\end{question}

%----------------------------------------------------------------------------------------

\end{document}
