%%%%%%%%%%%%%%%%%%%%%%%%%%%%%%%%%%%%%%%%%
% Cleese Assignment (For Students)
% LaTeX Template
% Version 2.0 (27/5/2018)
%
% This template originates from:
% http://www.LaTeXTemplates.com
%
% Author:
% Vel (vel@LaTeXTemplates.com)
%
% License:
% CC BY-NC-SA 3.0 (http://creativecommons.org/licenses/by-nc-sa/3.0/)
% 
%%%%%%%%%%%%%%%%%%%%%%%%%%%%%%%%%%%%%%%%%

%----------------------------------------------------------------------------------------
%	PACKAGES AND OTHER DOCUMENT CONFIGURATIONS
%----------------------------------------------------------------------------------------

\documentclass[11pt]{article}

\input{structure.tex} % Include the file specifying the document structure and custom commands
\usepackage{ctex}
\usepackage{tabularx}
\usepackage{booktabs}
\usepackage{graphicx}
\usepackage{caption}
\usepackage{subcaption}
\usepackage{float}
%----------------------------------------------------------------------------------------
%	ASSIGNMENT INFORMATION
%----------------------------------------------------------------------------------------

% Required
\newcommand{\assignmentQuestionName}{Problem} % The word to be used as a prefix to question numbers; example alternatives: Problem, Exercise
\newcommand{\assignmentClass}{ZJU Computational Physics} % Course/class
\newcommand{\assignmentTitle}{Homework\ \#5} % Assignment title or name
\newcommand{\assignmentAuthorName}{NAKO} % Student name

% Optional (comment lines to remove)
%\newcommand{\assignmentClassInstructor}{} % Intructor name/time/description
\newcommand{\assignmentDueDate}{Monday,\ December\ 23,\ 2024\\
Github: https://github.com/NAKONAKO4/ZJU-computational-physics-NAKO} % Due date
\usepackage{listings, xcolor}
\lstdefinestyle{lfonts}{
  basicstyle   = \footnotesize\ttfamily,
  stringstyle  = \color{purple},
  keywordstyle = \color{blue!60!black}\bfseries,
  commentstyle = \color{olive}\scshape,
}
\lstdefinestyle{lnumbers}{
  numbers     = left,
  numberstyle = \tiny,
  numbersep   = 1em,
  firstnumber = 1,
  stepnumber  = 1,
}
\lstdefinestyle{llayout}{
  breaklines       = true,
  tabsize          = 2,
  columns          = flexible,
}
\lstdefinestyle{lgeometry}{
  xleftmargin      = 20pt,
  xrightmargin     = 0pt,
  frame            = tb,
  framesep         = \fboxsep,
  framexleftmargin = 20pt,
}
\lstdefinestyle{lgeneral}{
  style = lfonts,
  style = lnumbers,
  style = llayout,
  style = lgeometry,
}
\lstdefinestyle{python}{
  language = {Python},
  style    = lgeneral,
}
%----------------------------------------------------------------------------------------

\begin{document}

%----------------------------------------------------------------------------------------
%	TITLE PAGE
%----------------------------------------------------------------------------------------

\maketitle % Print the title page

\thispagestyle{empty} % Suppress headers and footers on the title page

\newpage

%----------------------------------------------------------------------------------------
%	QUESTION 1
%----------------------------------------------------------------------------------------

\begin{question}

\questiontext{MD: Approach to Equilibrium}
\answer{a. 可以看到能量和动量都是守恒的,且动量基本为0($10^{-14}$数量级);同时计算得到的压强与理想气体压强差值相对误差较小。)
}
\begin{figure}[H]
  \centering
  \includegraphics[width=0.7\columnwidth]{1/a1.png}
  \caption{能量和动量随时间变化}
\end{figure}
\begin{figure}[H]
  \centering
  \includegraphics[width=0.7\columnwidth]{1/a2.png}
  \caption{温度和压强随时间变化}
\end{figure}

\begin{figure}[H]
  \centering
  \includegraphics[width=0.7\columnwidth]{1/a3.png}
  \caption{压强计算值与理想气体公式计算得到压强差值}
\end{figure}

\answer{b. 可以看到能量和动量都是守恒的,且动量基本为0($10^{-14}$数量级))
}
\begin{figure}[H]
  \centering
  \includegraphics[width=0.7\columnwidth]{1/b1.png}
  \caption{能量和动量随时间变化}
\end{figure}
\begin{figure}[H]
  \centering
  \includegraphics[width=0.7\columnwidth]{1/b2.png}
  \caption{温度和压强随时间变化}
\end{figure}

\answer{c. 可以看到接近平衡态时,左边粒子数均值会趋近于总粒子数的一半$N/2=9$)
}
\begin{figure}[H]
  \centering
  \includegraphics[width=0.7\columnwidth]{1/c1.png}
\end{figure}
\begin{figure}[H]
  \centering
  \includegraphics[width=0.7\columnwidth]{1/c2.png}
\end{figure}

\answer{d. 可以看到能量和动量都是守恒的,且动量基本为0($10^{-14}$数量级);同时计算得到的压强与理想气体压强差值相对误差较小。)
}
\begin{figure}[H]
  \centering
  \includegraphics[width=0.7\columnwidth]{1/d1.png}
  \caption{能量和动量随时间变化}
\end{figure}
\begin{figure}[H]
  \centering
  \includegraphics[width=0.7\columnwidth]{1/d2.png}
  \caption{温度和压强随时间变化}
\end{figure}

\begin{figure}[H]
  \centering
  \includegraphics[width=0.7\columnwidth]{1/d3.png}
  \caption{压强计算值与理想气体公式计算得到压强差值}
\end{figure}

\answer{e. 可以看到能量和动量都是守恒的,且动量基本为0($10^{-14}$数量级);同时计算得到的压强与理想气体压强差值相对误差较小。)
}
\begin{figure}[H]
  \centering
  \includegraphics[width=0.7\columnwidth]{1/e1.png}
  \caption{能量和动量随时间变化}
\end{figure}
\begin{figure}[H]
  \centering
  \includegraphics[width=0.7\columnwidth]{1/e2.png}
  \caption{温度和压强随时间变化}
\end{figure}

\begin{figure}[H]
  \centering
  \includegraphics[width=0.7\columnwidth]{1/e3.png}
  \caption{压强计算值与理想气体公式计算得到压强差值}
\end{figure}

\answer{abde题代码如下,对于不同的初始条件只需要更改代码中的初始条件设置即可。c题代码见本代码的后一部分}
\lstinputlisting[style = python]{1/mdmodal.py}
\answer{c题代码如下}
\lstinputlisting[style = python]{1/c.py}
\end{question}

%----------------------------------------------------------------------------------------
%	QUESTION 2
%----------------------------------------------------------------------------------------
\begin{question}
  \questiontext{Ising Model}
  \answer{a. 可以看到$T_c$随着材料扩大,逐渐逼近无限大系统的相变温度,大约在2.35附近}
  \begin{figure}[H]
    \centering
    \includegraphics[width=0.7\columnwidth]{2.png}
    \caption{L=2}
  \end{figure}
  \begin{figure}[H]
    \centering
    \includegraphics[width=0.7\columnwidth]{3.png}
    \caption{L=3}
  \end{figure}
  
  \begin{figure}[H]
    \centering
    \includegraphics[width=0.7\columnwidth]{4.png}
    \caption{L=4}
  \end{figure}
  \begin{figure}[H]
    \centering
    \includegraphics[width=0.7\columnwidth]{ising_L5.png}
    \caption{L=5}
  \end{figure}
  \lstinputlisting[style=python]{2/a.py}
  \answer{b. 可以看到$T_c$随着材料扩大逼近大约2.25附近的过程,其中L=128的图片由于材料较大,在有限计算资源下难以达到接近平衡态的状态,进而出现多个峰值,可以估计出真实峰值应该接近2.25。
  
  蒙特卡洛模拟参数:10000步MC来达到平衡态,之后5000步进行测量。由于L=64、L=128的材料尺寸很大,在这个MC模拟参数下运算一次需要很长时间,且计算资源有限,因此没有设置更大的模拟参数,如果计算资源足够可以设置更大参数,得到更准确结果。}
  \begin{figure}[H]
    \centering
    \includegraphics[width=0.7\columnwidth]{ising_model_L4.png}
    \caption{L=4}
  \end{figure}
  \begin{figure}[H]
    \centering
    \includegraphics[width=0.7\columnwidth]{ising_model_L6.png}
    \caption{L=6}
  \end{figure}
  
  \begin{figure}[H]
    \centering
    \includegraphics[width=0.7\columnwidth]{ising_model_L8.png}
    \caption{L=8}
  \end{figure}
  \begin{figure}[H]
    \centering
    \includegraphics[width=0.7\columnwidth]{ising_model_L16.png}
    \caption{L=16}
  \end{figure}
  \begin{figure}[H]
    \centering
    \includegraphics[width=0.7\columnwidth]{ising_model_L32.png}
    \caption{L=32}
  \end{figure}
  \begin{figure}[H]
    \centering
    \includegraphics[width=0.7\columnwidth]{ising_model_L64.png}
    \caption{L=64}
  \end{figure}
  
  \begin{figure}[H]
    \centering
    \includegraphics[width=0.7\columnwidth]{ising_model_L128.png}
    \caption{L=128}
  \end{figure}
\end{question}
\lstinputlisting[style=python]{2/b.py}

\end{document}