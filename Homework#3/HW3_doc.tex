%%%%%%%%%%%%%%%%%%%%%%%%%%%%%%%%%%%%%%%%%
% Cleese Assignment (For Students)
% LaTeX Template
% Version 2.0 (27/5/2018)
%
% This template originates from:
% http://www.LaTeXTemplates.com
%
% Author:
% Vel (vel@LaTeXTemplates.com)
%
% License:
% CC BY-NC-SA 3.0 (http://creativecommons.org/licenses/by-nc-sa/3.0/)
% 
%%%%%%%%%%%%%%%%%%%%%%%%%%%%%%%%%%%%%%%%%

%----------------------------------------------------------------------------------------
%	PACKAGES AND OTHER DOCUMENT CONFIGURATIONS
%----------------------------------------------------------------------------------------

\documentclass[11pt]{article}

\input{structure.tex} % Include the file specifying the document structure and custom commands
\usepackage{ctex}
\usepackage{tabularx}
\usepackage{booktabs}
\usepackage{graphicx}
\usepackage{caption}
\usepackage{subcaption}
\usepackage{float}
%----------------------------------------------------------------------------------------
%	ASSIGNMENT INFORMATION
%----------------------------------------------------------------------------------------

% Required
\newcommand{\assignmentQuestionName}{Problem} % The word to be used as a prefix to question numbers; example alternatives: Problem, Exercise
\newcommand{\assignmentClass}{ZJU Computational Physics} % Course/class
\newcommand{\assignmentTitle}{Homework\ \#3} % Assignment title or name
\newcommand{\assignmentAuthorName}{NAKO} % Student name

% Optional (comment lines to remove)
%\newcommand{\assignmentClassInstructor}{} % Intructor name/time/description
\newcommand{\assignmentDueDate}{Monday,\ December\ 9,\ 2024\\
Github: https://github.com/NAKONAKO4/ZJU-computational-physics-NAKO} % Due date
\usepackage{listings, xcolor}
\lstdefinestyle{lfonts}{
  basicstyle   = \footnotesize\ttfamily,
  stringstyle  = \color{purple},
  keywordstyle = \color{blue!60!black}\bfseries,
  commentstyle = \color{olive}\scshape,
}
\lstdefinestyle{lnumbers}{
  numbers     = left,
  numberstyle = \tiny,
  numbersep   = 1em,
  firstnumber = 1,
  stepnumber  = 1,
}
\lstdefinestyle{llayout}{
  breaklines       = true,
  tabsize          = 2,
  columns          = flexible,
}
\lstdefinestyle{lgeometry}{
  xleftmargin      = 20pt,
  xrightmargin     = 0pt,
  frame            = tb,
  framesep         = \fboxsep,
  framexleftmargin = 20pt,
}
\lstdefinestyle{lgeneral}{
  style = lfonts,
  style = lnumbers,
  style = llayout,
  style = lgeometry,
}
\lstdefinestyle{python}{
  language = {Python},
  style    = lgeneral,
}
%----------------------------------------------------------------------------------------

\begin{document}

%----------------------------------------------------------------------------------------
%	TITLE PAGE
%----------------------------------------------------------------------------------------

\maketitle % Print the title page

\thispagestyle{empty} % Suppress headers and footers on the title page

\newpage

%----------------------------------------------------------------------------------------
%	QUESTION 1
%----------------------------------------------------------------------------------------

\begin{question}

\questiontext{Root-finding}
\answer{a. a=1.09和a=0.21分别绘图可得:
		\begin{center}
			\includegraphics[width=\columnwidth]{1_1.png}
		\end{center}
		\begin{center}
			\includegraphics[width=\columnwidth]{1_2.png}
		\end{center}

		可得在a=1.09时只有一根,大约为0.7;在a=0.21时有3根,大约为-3.7、-2.0、1.3。
}
\answer{b-f. 运行结果为
		\begin{center}
			\includegraphics[width=\columnwidth]{1_3.png}
		\end{center}
		}
\lstinputlisting[style = python]{1.py}
\end{question}

%----------------------------------------------------------------------------------------
%	QUESTION 2
%----------------------------------------------------------------------------------------

\begin{question}

\questiontext{Bifurcation Diagram}
\lstinputlisting[style = python]{2.py}
\answer{运行结果为:

\includegraphics[width=\columnwidth]{2.png}}
%--------------------------------------------
%--------------------------------------------

\end{question}
\end{document}
